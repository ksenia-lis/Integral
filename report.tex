\documentclass[a4paper,12pt,titlepage,finall]{article}

\usepackage[T1,T2A]{fontenc}     % форматы шрифтов
\usepackage[utf8x]{inputenc}     % кодировка символов, используемая в данном файле
\usepackage[russian]{babel}      % пакет русификации
\usepackage{tikz}                % для создания иллюстраций
\usepackage{pgfplots}            % для вывода графиков функций
\usepackage{geometry}		 % для настройки размера полей
\usepackage{indentfirst}         % для отступа в первом абзаце секции
\usepackage{pb-diagram}

% выбираем размер листа А4, все поля ставим по 3см
\geometry{a4paper,left=30mm,top=30mm,bottom=30mm,right=30mm}

\setcounter{secnumdepth}{0}      % отключаем нумерацию секций

\usepgfplotslibrary{fillbetween} % для изображения областей на графиках

\begin{document}
% Титульный лист
\begin{titlepage}
    \begin{center}
	{\small \sc Московский государственный университет \\имени М.~В.~Ломоносова\\
	Факультет вычислительной математики и кибернетики\\}
	\vfill
	{\Large \sc Отчет по заданию №6}\\
	~\\
	{\large \bf <<Сборка многомодульных программ. \\
	Вычисление корней уравнений и определенных интегралов.>>}\\ 
	~\\
	{\large \bf Вариант 7 / 2 / 2}
    \end{center}
    \begin{flushright}
	\vfill {Выполнил:\\
	студентка 101 группы\\
	Лисицина~К.~А.\\
	~\\
	Преподаватель:\\
	Кузьменкова~Е.~А.}
    \end{flushright}
    \begin{center}
	\vfill
	{\small Москва\\2016}
    \end{center}
\end{titlepage}

% Автоматически генерируем оглавление на отдельной странице
\tableofcontents
\newpage

\section{Постановка задачи}
\begin{itemize}
\item Требуется реализовать численный метод, позволяющий вычислять площадь плоской фигуры, ограниченной тремя кривыми, путём нахождения точек пересечения кривых и вычисления площади под графиками кривых на соответствующих отрезках:
\item Площадь под графиком необходимо искать квадратурной формулой трапеций.
\item Вершины фигуры необходимо искать методом хорд.
\item Отрезок для применения метода нахождения корней должен быть вычислен аналитически.
\item Требуемая точность вычисления площади $\varepsilon$= 0.001.
\end{itemize}

\newpage

\section{Математическое обоснование}

Пусть искомый корень уравнения f(x) = 0 изолирован на некотором сегменте [a, b]. Предположим, что данная функция имеет на этом сегменте монотонную и непрерывную производную, сохраняющую опрееделенный знак. При этом возможны четыре случая:\\
1. f'(x) не убывает и положительна на [a, b], \\
2. f'(x) не возрастает и отрицательна на [a, b], \\
3. f'(x) не возрастает и положительна на [a, b], \\
4. f'(x) не убывает и отрицательна на [a, b].
\begin{itemize}
\item В  1, 2 случаях справедлива индукционная формула {$x_{n+1} = x_n - \frac{(b - x_n)f(x_n)}{f(b)-f(x_n)}$}
\item В 3, 4 случаях справедлива индукционная формула  {$x_{n+1} = x_n - \frac{(a - x_n)f(x_n)}{f(a)-f(x_n)}$}
\end{itemize}
Причем {$\lim_{n \to \infty} x_n = c$}, где f(c) = 0. Такая последовательность будет сходиться к корню, значит можно получить значение с какой угодно точностью.(метод хорд~\cite{math})\\
Заданые функции приведены ниже (рис.~\ref{plot1})

\begin{figure}[h]
\centering
\begin{tikzpicture}
\begin{axis}[% grid=both,                % рисуем координатную сетку (если нужно)
             axis lines=middle,          % рисуем оси координат в привычном для математики месте
             restrict x to domain=-2:10,  % задаем диапазон значений переменной x
             restrict y to domain=-2:10,  % задаем диапазон значений функции y(x)
             axis equal,                 % требуем соблюдения пропорций по осям x и y
             enlargelimits,              % разрешаем при необходимости увеличивать диапазоны переменных
             legend cell align=left,     % задаем выравнивание в рамке обозначений
             scale=2]                    % задаем масштаб 2:1

% первая функция
% параметр samples отвечает за качество прорисовки
\addplot[green,domain=-2:9, samples=256,thick] {ln(x)};
% описание первой функции
\addlegendentry{$y=ln(x)$}

% добавим немного пустого места между описанием первой и второй функций
\addlegendimage{empty legend}\addlegendentry{}

% вторая функция
% здесь необходимо дополнительно ограничить диапазон значений переменной x
\addplot[blue,domain=-2:9,samples=256,thick] {-2*x + 14};
\addlegendentry{$y=(-2)*x + 14$}

% дополнительное пустое место не требуется, так как формулы имеют небольшой размер по высоте

% третья функция
\addplot[red,domain=2:9,samples=256,thick] {1/(2-x) + 6};
\addlegendentry{$y=\frac{1}{2-x} + 6$}
 
% добавим немного пустого места между описанием первой и второй функций
\addlegendimage{empty legend}\addlegendentry{}

\end{axis}
\end{tikzpicture}
\caption{Плоская фигура, ограниченная графиками заданных уравнений}
\label{plot1}
\end{figure}

\newpage
\section{Выбор отрезков}
\begin{itemize}
\item $g_{12} = f_1 - f_2$ корень ищется на отрезке [5, 8]:\\
$g_{12}(5)=1.6 - 4 < 0$, $g_{12}(8)= 2.07 - 2 < 0$\\
$g_{12}'(x) = \frac{1}{x} + 2 $ при $x \in{[5, 8]}$ > 0\\
$g_{12}''(x) = -\frac{1}{x^2}$ при $x \in{[5, 8]}$ < 0\\
Условия выбора отрезка выполнены.
\item  $g_{23} = f_2 - f_3$ корень ищется на отрезке [4, 5]:\\
$g_{23}(4)=5.5 - 6 < 0$, $g_{23}(5)= 5.7 - 4 > 0$\\
$g_{23}'(x) = -2 - \frac{1}{(2-x)^2}$ при $x \in{[4, 5]}$ < 0\\
$g_{23}''(x) = \frac{4-2x}{(2-x)^4}$ при $x \in{[4, 5]}$ < 0\\
Условия выбора отрезка выполнены.
\item  $g_{12} = f_1 - f_3$ корень ищется на отрезке [2.1, 3]:\\
$g_{13}(2.1)= 0.741 + 4 > 0$, $g_{13}(3)= 1.1 - 5 < 0$\\
$g_{13}'(x) =  \frac{1}{x} - \frac{1}{(2-x)^2}$ при $x \in{[2.1, 3]}$ < 0\\
$g_{13}''(x) = -\frac{1}{x^2} - \frac{4-2x}{(2-x)^4}$ при $x \in{[2.1, 3]}$ < 0\\
Условия выбора отрезка выполнены.
\end{itemize}
\section{Выбор $\varepsilon_1$  и $\varepsilon_2$.} 

Все значения функций на заданных отрезках не превосходят 10, значит, можно сказать, что итоговая погрешность вычисления площади под графиком $\varepsilon_{finally}\leq$ $\varepsilon_2 + 10*\varepsilon_1$,\\ значит итоговая погрешность вычисляется следующим образом\\ $\varepsilon = 3*\varepsilon_{finally} = 3*(\varepsilon_2 + 10*\varepsilon_1) = 0.001$\\
$\varepsilon$ - общая погрешность вычисления площади.\\
$\varepsilon_1$ - заданная погрешность вычисления корня.\\
$\varepsilon_2$ - заданная погрешность вычисления площади под графиком.\\
Отсюда, $\varepsilon_1$ можно взять 0.00001, а $\varepsilon_2 = 0.0001$.
\newpage
\section{Результаты экспериментов}
\section{Результаты вычислений}
Все полученные значения представлеты в таблице(таблица~\ref{table1})

\begin{table}[h]
\centering
\begin{tabular}{|c|c|c|}
\hline
Кривые & $x$ & $y$ \\
\hline
1 и 2 & 6.096 & 1.808 \\
2 и 3 & 4.225 & 5.551\\
1 и 3 & 2.192 & 0.786 \\
\hline 
\end{tabular}
\caption{Координаты точек пересечения}
\label{table1}
\end{table}


\section{Графическое представление}

\begin{figure}[h]
\centering
\begin{tikzpicture}
\begin{axis}[% grid=both,                % рисуем координатную сетку (если нужно)
             axis lines=middle,          % рисуем оси координат в привычном для математики месте
             restrict x to domain=-2:10,  % задаем диапазон значений переменной x
             restrict y to domain=-2:10,  % задаем диапазон значений функции y(x)
             axis equal,                 % требуем соблюдения пропорций по осям x и y
             enlargelimits,              % разрешаем при необходимости увеличивать диапазоны переменных
             legend cell align=left,     % задаем выравнивание в рамке обозначений
             scale=2,                    % задаем масштаб 2:1
             xticklabels={,,},           % убираем нумерацию с оси x
             yticklabels={,,}]           % убираем нумерацию с оси y

% первая функция
% параметр samples отвечает за качество прорисовки
\addplot[green,domain=-2:10, samples=256,thick,name path=A] {ln(x)};
% описание первой функции
\addlegendentry{$y=ln(x)$}

% вторая функция
% здесь необходимо дополнительно ограничить диапазон значений переменной x
\addplot[blue,domain=-2:10,samples=256,thick,name path=B] {-2*x + 14};
\addlegendentry{$y=(-2)*x + 14$}

% дополнительное пустое место не требуется, так как формулы имеют небольшой размер по высоте

% третья функция
\addplot[red,domain=2:10, samples=256,thick,name path=C] {1/(2-x) + 6};
\addlegendentry{$y=\frac{1}{2-x} + 6$}

% добавим немного пустого места между описанием первой и второй функций
\addlegendimage{empty legend}\addlegendentry{}


% закрашиваем фигуру
\addplot[blue!20,samples=256] fill between[of=C and A,soft clip={domain=2.177:4.225}];
\addplot[blue!20,samples=256] fill between[of=A and B,soft clip={domain=4.1:6.2}];
\addlegendentry{$S=11.236$}

% Поскольку автоматическое вычисление точек пересечения кривых в TiKZ реализовать сложно,
% будем явно задавать координаты.
\addplot[dashed] coordinates { (4.225, 5.551) (4.225, 0) };
\addplot[color=black] coordinates {(4.225, 0)} node [label={-10:{\small 4.225}}]{};

\addplot[dashed] coordinates { (2.192, 0.786) (2.192, 0) };
\addplot[color=black] coordinates {(2.192, 0)} node [label={-10:{\small 2.192}}]{};

\addplot[dashed] coordinates { (6.096, 1.808) (6.096, 0) };
\addplot[color=black] coordinates {(6.096, 0)} node [label={-10:{\small 6.096}}]{};

\end{axis}
\end{tikzpicture}
\caption{Плоская фигура, ограниченная графиками заданных уравнений}
\label{plot2}
\end{figure}

\newpage

\section{Структура программы и спецификация функций}
\section{Модуль asmintegral.asm:}
\begin{itemize}
\item float f1(float x); возвращает значение ln(x)
\item float f2(float x); возвращает значение -2*х + 14
\item float f3(float x); возвращает значение 1/(2 − x)+6
\item float f4(float x); возвращает значение 3/(х-4) + 4
\item float f5(float x); возвращает значение 3/х
\end{itemize}
\section{Модуль integral.с:}
\begin{itemize}
\item float root(float(*f)(float), float(*g)(float), float a, float b, float eps1); вычисляет точку пересечения функций f и g на отрезке[a,b]с точностью eps1, используя метод хорд и производные функций f и g.
\item float integral(float(*f)(float), float a, float b, float eps2); вычисляет площадь под графиком функции f на отрезке [a, b] с точностью eps2,
используя метод трапеций.
\item int main(int argc, char ** argv); функция main
\end{itemize}

\newpage

\section{Сборка программы (Make-файл)}
\section{Makefile}
\begin{itemize}
\item all: integral
\item asmintegral.o: asmintegral.asm
 \\nasm -f elf32 -o asmintegral.o asmintegral.asm
\item integral: integral.o asmintegral.o
    \\ gcc -m32 -o integral integral.o asmintegral.o
\item integral.o: integral.c 
	\\gcc -m32 -std=c11 -c -o integral.o integral.c
\item clean:
	\\rm *.o
\end{itemize}	

\[ \begin{diagram}
   \node{asmintegral.asm}\arrow[2]{t,e}
   \node[2]{asmintegral.o}\arrow[2]{se,b}\\\\\\
   \node{integral.c}\arrow[2]{t,e}
   \node[2]{integral.o}\arrow[2]{t,e}
   \node[2]{integral}\\
\end{diagram}\]               


\newpage

\section{Отладка программы, тестирование функций}

Тестирование численных методов провидилось на тестовых функциях\\ $f_4 = \frac{3}{x-4} + 4$ и $f_5 = \frac{3}{x}$ c использованием их производных \\ $f_4'(x) = -\frac{3}{(x-4)^2}$, $f_5'(x)=-\frac{3}{x^2}$,\\ $f_4''(x) =\frac {6}{(x-4)^3}$, $f_5''(x) = \frac{6}{x^3}$
\section{Тесты функции root}
\begin{enumerate}
\item float $x_4$ = root($f_4, f_5, 0.5, 1.02, 0.0001$) = 1; функции $f_4$ и $f_5$ пересекаются в точке с х = 1.\\
$(f_4 - f_5)(0.5) = 3.14 - 6 < 0$, $(f_4 - f_5)(1.02) = 2.492 - 1.492 > 0$ 
$(f_4 - f_5)'(x)$ всегда < 0\\
$(f_4 - f_5)''(0.5) > 0$ и $(f_4 - f_5)''(1.02) > 0$\\
Условия выполнены.
\item float $x_5$ = root($f_4, f_5$, 2.5, 3.5, 0.0001) = 3; функции $f_4$ и $f_5$ пересекаются в точке с х = 3.\\
$(f_4 - f_5)(2.5) = 2 - 1.2 > 0$, $(f_4 - f_5)(3.5) = -2 - 0.8 < 0$ 
$(f_4 - f_5)'(x)$ всегда < 0\\
$(f_4 - f_5)''(2.5)$ < 0 и $(f_4 - f_5)''(3.5)$ < 0\\
Условия выполнены.
\end{enumerate}

\section{Тесты функции integral}
\begin{enumerate}
\item integral(f4, x4, x5, 0.0001) = 4.704; $\int_{1}^{3}(\frac{3}{x - 4} + 4)dx = 4x + 3*log(x - 4) |_{1}^{3} = 8 - 3*log(3)$ = 4.702
\item integral(f5, x4, x5, 0.0001) = 3.296
$\int_{1}^{3}\frac{3}{x}dx = 3*log(x)|_{1}^{3} = log(27)$ = 3.296
\end{enumerate}
\newpage

\section{Программа на Си и на Ассемблере}
Исходные тексты программ на си и ассемблере имеются в сданном архиве.
\newpage
\begin{raggedright}
\addcontentsline{toc}{section}{Список цитируемой литературы}
\begin{thebibliography}{99}
\bibitem{math} Ильин~В.\,А., Садовничий~В.\,А., Сендов~Бл.\,Х. Математический анализ. Т.\,1~---
    Москва: Наука, 1985.
\end{thebibliography}
\end{raggedright}

\end{document}
